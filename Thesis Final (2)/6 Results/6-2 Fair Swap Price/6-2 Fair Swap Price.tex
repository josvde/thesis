
\section{Fair Swap Prices at Contract Initiation}


From equation 2.1(1), we can find $K$ through:


     
\begin{equation}
    \mathbb{E}^\mathbb{Q} [K^2] = K^2 = \mathbb{E}^\mathbb{Q} [min(\sigma_{R_{Y}}^2,2,5K)-K^2]
    \label{eq:fair-swap-K}
\end{equation}\\

Note that we use the sum the daily variances $\sigma_{R_{D}}$ to find the fair swap rate on yearly realized variances, by minimizing the expression below

--------------------\\
\textit{add definition g(K) here}\\
----------------------\\

The method to find the fair strike value based on the simulated daily realized variances, is an iterative approach where a Newton-Raphson optimization scheme is implemented. \\

\textbf{Newton Raphson}\\

Denote $\tilde{K}$ as the Fair Swap Rate for a Capped Volatility Swap then $K=\tilde{K}^2$ becomes the Fair Swap Rate for a Variance Swap.

From a vector of realized variances of size $n$, $K$ is calculated iteratively by minimizing the formula that gives the risk-neutral price, $K$ satisfies the following expression:

\begin{equation}
g(K) < \epsilon
%\label{eq:foobar} %add label if required here
\end{equation}
where $g(K) = \mathbb{E}(min(\sigma_{R}^2; 2.5K)) - K$, and $\epsilon$ a small number which serves as a stopping ceriterion in the iterative method.

For $i=1,...,g-1$ \hspace{0.2cm} $g(K_i)>\epsilon$ and $g(K_2)<\epsilon$ then $K_\theta=K$ where we calculate these values by:

\begin{equation}
K_{i+1} = K_i - \frac{g(K_i)}{g'(K_i)} 
%\label{eq:foobar} %add label if required here
\end{equation} 

where\\
-----------\\
\textit{update g(K) here}\\
-----------------\\

and $g'(K_i)$ is given by:
 
\begin{equation}
g'(K_i) = \frac{[g(K_i) - g(K_{i-1})]}{K_i - K_{i-1}}
%\label{eq:foobar} %add label if required here
\end{equation} 

For large values of $n$, above $10^6$, $g$ remained relatively small and did not exceed $15$ in most cases.
