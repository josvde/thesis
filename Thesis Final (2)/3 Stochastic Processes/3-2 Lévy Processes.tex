
\section{Lévy Processes}

- short definition, talk about jumps...\\
- mention that VG and NIG processes are solid alternatives to BM (see beginning C5), because they allow skewness and excess curtosis.\\

Defining Lévy processes, focussing on finite vs. infite Jump processes. 

\subsection{Poisson and Gamma processes}

- 2 defs Poisson \\
- Short def Gamma\\

For Poisson: give two definitions a PP, i.i.d. stationary poisson incremenst (see book Jan De Spiegeleer). Then focus on the definition with exponential inter-arrival times. Link this with the fact that we need this to model Jumps.\\

For Gamma: explain why Gamma(a,b) equals Gamma(a,1) / b. And how the Gamma distribution is derived from the exponential distribution.\\

Explain how the Lévy triplet of a Gamma process indicates that it has an infinite number of jumps continuously through time. This is an important characteristic when using a Gamma process to model jumps.\\

\textbf{Definition 1:} A Poisson Process as a Counting Process with stationary independent increments

A counting process $N=\{N_t, t\geq0\}$ is a Poisson process with intensity parameter $\lambda$, $i\sigma$:
\begin{enumerate}
    \item The increments are stationary for $\mathscr{L}$, $s$ and $\Delta t\geq0:$\\
    $P(N_{t+\Delta t} - N_t = \sigma) = P (N_{s+\Delta t} -N_s = \mathscr{L}$)
    \item Additionally the increments are independent: $\forall 0\leq t_e \leq \ldots \leq t_\mathscr{L}$:\\
    $N(t_1)-N(t_0)$, $N(t_2)-N(t_1)$, $\ldots$ , $N(\mathscr{L}) - N(\mathscr{L}-1)$\\
    and follow a Poisson distribution with rate $\lambda$.
\end{enumerate}
In Chapter 5, we will use this definition of the Poisson process in the Uniform method for similarities.\\

\textbf{Exponential Spacings Method}\\
Let $T_1$, $T_2$, $\ldots$ be i.i.d. Exponential ($\lambda$) r.vs. and let
\begin{equation}
t_{n} = \sum_{k=1}^{n} T_k \hspace{0.5cm}t_0=0
%\label{eq:foobar} %add label if required here
\end{equation} 
Then $N(t) = max\{n: t_n \leq t\}$ is a Poisson process with rate $\lambda$
\begin{itemize}
    \item $t_n$: time of arrival of the $n$-th event
    \item $T_n$: time between the $(n-1)$th and the $n$-th arrival ($t_n - t_{n-1}$)
\end{itemize}



\subsection{Compound Poisson process}
- general def\\
- mention some good jump-size distribution (exponetial, gamma,...), and say we will use Gamma jumps in subsequent processes.\\

A Compound Poisson is defined by...\\

\begin{equation}
\sum_{k=N_{(n-1)\Delta t}+1}^{N_{n\Delta t}} x_{k}
\label{eq:5-2 Gamma-OU BDLP Compound Poisson Approximation}
\end{equation}\\
\td{Finish the discussion on Compound Poisson processes by explaining how they can be used to approximate BDLP processes.}



\subsection{Variance Gamma process}
- Also 1 short def\\

Use formulas simulation methods C5 to quickly set up defs\\

