
\section{Geometric Brownian Motion}


\begin{itemize}
	\item 3.2.1 definition; say its a Lévy
	\item path properties: infinite variation each interval
	\item scaling propery $ cW_{t} (\sigma=1) = W_{t} (\sigma = c)$
\end{itemize}



As a stochastic process is standard Brownian Motion $W = \{ W_{t}, t \geq 0 \} $ with $W_{0} = 0$ if it has the following properties:

\begin{itemize}

\item[(i)] it has indipendent increments
\item[(ii)] it has stationary increments
\item[(iii)] $W_{t+s} - W_{t} \sim$ Normal (0,s)

\end{itemize}

Brownian Motion is an example of a Lévy Process (see infra).\\
Additionally, time-interval of all possible sizes in paths of Brownian Motion have variation proportionally to the interval size, i.e. the increments are normally distributed with a variance proportional to the interval size.\\

\textbf{Geometric Brownian Motion}
Black and Scholes proposed in 1973 to use Geometric Brownian
Motion (GBM) as a continuous time stochastic process to model
stock prices, by the SDE given below:
\begin{equation}
d\hspace{0.1em}S_{t}=\mu S_{t}dt + \sigma S_{t}\hspace{0.1em}
d\hspace{0.15em} W_{t}
\label{eq:BS1}
\end{equation}

From this they derived a process for the natural logarithm of
the stock price:

\begin{equation}
d\hspace{0.1em}Z_{t}=d\hspace{0.1em}ln(S_{t})=(\mu -
\frac{1}{2}\sigma^{2})dt + \sigma d\hspace{0.15em} W_{t}
\label{eq:BS2}
\end{equation}

where $W = \{W_{t},t\geq0\}$ represents standard Brownian Motionin both versions of the GBM.\\

\textbf{Need for Lévy process to extend on Normally distributed
increments in Brownian Motion}\\
