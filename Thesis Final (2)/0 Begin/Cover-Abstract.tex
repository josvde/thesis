% ----------------------- Cover --------------------------------------
% Please fill in:% - The title and subtitle (if applicable)
%         to include a formula in the title or subtitle
%         use  \form{$...$}
% - Your name
% - Your (co)supervisor, mentor (if applicable)
% - Your master
% - The academic year
% --------------------------------------------------------------------
\thispagestyle{empty}
\newcommand{\form}[1]{\scalebox{1.087}{\boldmath{#1}}}
\sffamily
%
\begin{textblock}{191}(-24,-11)
\colorbox{bluetitle}{\hspace{95mm}\ \parbox[c][18truemm]{94mm}{\textcolor{white}{FACULTY OF ECONOMICS AND BUSINESS}}}
\end{textblock}
%
\begin{textblock}{70}(-18,-19)
\textblockcolour{}
\includegraphics*[height=19.8truemm]{9 MainFormat/LogoKULeuven}
\end{textblock}
%
\begin{textblock}{160}(-6,50)
\textblockcolour{}
\vspace{-\parskip}
\flushleft
\fontsize{40}{42}\selectfont \textcolor{bluetitle}{Stochastic Volatility Models for Pricing Volatility Swaps}\\[1.5mm]
\fontsize{20}{22}\selectfont Addressing Model Selection Risk in Pricing with Monte Carlo Methods
\end{textblock}
%
\begin{textblock}{160}(15,153)
\textblockcolour{}
\vspace{-\parskip}
\flushright
\fontsize{14}{16}\selectfont \textbf{Jos Van den Eynde}\\[6pt]
\fontsize{8}{10}\selectfont R0758751
\end{textblock}
%
\begin{textblock}{160}(15,173)
\textblockcolour{}
\vspace{-\parskip}
\flushright
Thesis submitted to obtain the degree of\\[4.5pt]
MASTER OF ACTUARIAL AND FINANCIAL ENGINEERING\\[4.5pt]
\end{textblock}
%
\begin{textblock}{160}(15,215)
\textblockcolour{}
\vspace{-\parskip}
\flushright
Promoter: Prof. Dr. Wim Schoutens\\[4.5pt]
Academic year 2023-2024
\end{textblock}
%
\begin{textblock}{191}(-24,248)
{\color{blueline}\rule{550pt}{5.5pt}}
\end{textblock}
%
\vfill
\newpage

\thispagestyle{empty}


\setlength{\parindent}{0pt}
\makeatletter
\renewcommand{\@arrayparboxrestore}{\raggedright\normalfont}
\makeatother

\begin{textblock}{191}(-24,-11)
\colorbox{bluetitle}{\hspace{95mm}\ \parbox[c][18truemm]{94mm}{\textcolor{white}{FACULTY OF ECONOMICS AND BUSINESS}}}
\end{textblock}
%
\begin{textblock}{70}(-18,-19)
\textblockcolour{}
\includegraphics*[height=19.8truemm]{9 MainFormat/LogoKULeuven}
\end{textblock}
%
\begin{textblock}{160}(-6,50)
\textblockcolour{}
\vspace{-\parskip}
\flushleft
\fontsize{40}{42}\selectfont \textcolor{bluetitle}{Stochastic Volatility Models for Pricing Volatility Swaps}\\[1.5mm]
\fontsize{20}{22}\selectfont Addressing Model Selection Risk in Pricing with Monte Carlo Methods\\[5.5mm]
\fontsize{10}{12}\selectfont \justify
In this thesis, we compare the pricing of Volatility Swaps (VOSWs) using different Stochastic Volatility (SV) models, where the price is modeled by Stochastic Differential Equations (SDEs) employing various stochastic processes. Sometimes it is necessary to derive a process for the natural logarithm of the price, referred to as the log-price process. For the pricing of VOSWs, we need the variance on the returns, not the variance of the prices. Consequently, we use log-returns to calculate the realized variance, and this can be done from both the price process and the log-price process. \\

Subsequently, we will investigate pricing methods with the Heston model, Barndorff-Nielsen and Shephard (BNS) model, and the Lévy Stochastic Time Change model. These SV models all incorporate SV differently, either by means of a variance process or alternatively by means of a stochastic process for the rate of time change. The latter incorporates stochastic volatility indirectly by making time stochastic within other processes of the price SDE. These stochastic time dynamics are a solid alternative to the use of variance processes. For each model, we formally describe the reasoning behind the underlying dynamics that are modeled by extensively discussing the SDEs and stochastic processes within each model. \\

Moreover, we cover in more detail all simulation techniques needed to draw random paths for the price processes by explaining techniques to generate random draws from varying probability density functions. As an example, in the variance process under BNS, an Ornstein-Uhlenbeck (OU) process is used, where a Gamma density is assigned to the Background Driving Lévy Process (BDLP). This results in the need for random draws from the Poisson, Exponential, Standard Normal, and Standard Uniform distributions to simulate paths for the log-price process. \\

Finally, we will compare final price differences between models and discuss how these findings can be explained by the characteristics of the different SDEs employed, and conclude by critically reflect on the use of Monte Carlo methods for the price process simulation.
\end{textblock}

\newpage
%
% In case you want to integrate the TeX-file for the titlepage
% with the rest of your thesis, you can continue below