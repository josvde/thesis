Stochastic processes are used to model the behavior of random variables over time in various fields such as applied mathematics, finance and engineering to name a few. In a stochastic process, the future behavior of the system is not completely determined by its current state, but is influenced by randomness or uncertainty. Besides temporal stochastic processes, spatial stochastic processes exist to model randomness occurring in areas or volumes. Examples of spatial processes can be found in meteorology for weather phenomena across geographical locations. In quantitative finance, almost all processes will model movements of financial markets over time.\\ 

A collection of random variables over an indexed parameter, like time, is a common definition for a stochastic process. A distinction is made between discrete and continuous stochastic processes, where a discrete-time stochastic process can be represented as $X=\{X_n\}_{n\in\mathbb{N}}$, where $X_n$ represents the random variable at discrete time point $n$ in the stochastic process $X$. On the other hand, a stochastic process in continuous time will be represented as $Y=\{Y_{t},t\geq0\}$, where $Y_{t}$ represents the random variable at time $t$ of the stochastic process $Y$.\\

There are various types of stochastic processes, including discrete-time stochastic processes like Markov chains and random walks, as well as continuous-time stochastic processes like Brownian motion. Stochastic processes can also be classified as stationary, where the statistical properties do not change over time, or non-stationary, where the statistical properties change over time.\\

Stochastic processes are used in many applications, including modeling stock prices and financial markets, predicting traffic flow and queueing systems, understanding population dynamics in biology, and analyzing random signals in engineering.\\

To summarize, stochastic processes provide a powerful framework for modeling and analyzing random phenomena. By understanding the basic concepts and types of stochastic processes, we can apply them to solve real-world problems in various fields. We now turn our attention to some of the key types of stochastic processes that are essential in various applications and models discussed in this thesis.\\

A first important process is Brownian Motion which is a fundamental stochastic process characterized by its random, continuous path that evolves over time. It is found in Geometric Brownian Motion, which is the process that solves the Black-Scholes Stochastic Differential Equation (SDE). Additionally, Brownian Motion is used in the Cox-Ingersoll-Ross (CIR) process, which is a crucial component of the Heston model. The distinctive feature of Brownian Motion is that it has normally distributed changes, meaning that over any small time interval, the change in its value follows a normal distribution with an average change of zero and a variability that depends on the length of the interval. This property makes Brownian Motion a foundational building block in the modeling of random behavior in markets.\\

Brownian Motion is also an example of a Lévy process, a broader class of stochastic processes that include jumps or discontinuities. Lévy processes have the important characteristic of having independent and identically distributed increments, meaning the changes over non-overlapping time intervals are statistically independent and follow the same probability distribution. This class of processes can capture more complex behaviors in financial markets, such as sudden jumps in prices.\\

Within the realm of Lévy processes are specific examples like the Inverse Gamma Lévy and the Variance Gamma Lévy processes. These processes model phenomena that cannot be adequately captured by Brownian Motion alone, incorporating heavier tails and greater variability. Understanding these processes provides deeper insights into market dynamics, particularly in scenarios involving abrupt changes or high volatility.\\

Another important type of process is the Ornstein-Uhlenbeck (OU) process, which is particularly useful for modeling mean-reverting behavior. This means that the process tends to move back towards a long-term average over time. The OU process is widely applied in modeling interest rates, volatility, and other financial quantities that exhibit this kind of reverting behavior. When the driving force behind the OU process is a Lévy process, such as the Inverse Gamma process, even more complex and realistic market behaviors can be modeled, capturing both the mean-reverting nature and the possibility of sudden jumps.\\

These processes form a comprehensive toolkit for modeling various types of random behavior in financial markets. Each has unique properties that make it suitable for different applications, from the smooth, continuous paths of Brownian Motion to the jump-diffusion characteristics of Lévy processes and the mean-reverting nature of Ornstein-Uhlenbeck processes. This overview sets the stage for a more detailed and formal discussion of these processes, which will follow in the subsequent parts of the thesis.